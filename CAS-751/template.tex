\documentclass[11pt]{article}

% basic packages
\usepackage[margin=1in]{geometry}
\usepackage[pdftex]{graphicx}
\usepackage{amsmath,amssymb,amsthm}
\usepackage{custom}
\usepackage{lipsum}

% page formatting
\usepackage{fancyhdr}
\pagestyle{fancy}

\renewcommand{\sectionmark}[1]{\markright{\textsf{\arabic{section}. #1}}}
\renewcommand{\subsectionmark}[1]{}
\lhead{\textbf{\thepage} \ \ \nouppercase{\rightmark}}
\chead{}
\rhead{}
\lfoot{}
\cfoot{}
\rfoot{}
\setlength{\headheight}{14pt}

\linespread{1.03} % give a little extra room
\setlength{\parindent}{0.2in} % reduce paragraph indent a bit
\setcounter{secnumdepth}{2} % no numbered subsubsections
\setcounter{tocdepth}{2} % no subsubsections in ToC

\begin{document}

% make title page
\thispagestyle{empty}
\bigskip \
\vspace{0.1cm}

\begin{center}
{\fontsize{22}{22} \selectfont Lecture Notes on}
\vskip 16pt
{\fontsize{36}{36} \selectfont \bf \sffamily Group Theory for Optimization and Machine Learning}
\vskip 24pt
{\fontsize{18}{18} \selectfont \rmfamily Dhilan Teeluckdharry} 
\vskip 6pt
{\fontsize{14}{14} \selectfont \ttfamily teeluckn@mcmaster.ca} 
\vskip 24pt
\end{center}


% {\parindent0pt \baselineskip=15.5pt \lipsum[1-4]}

% make table of contents
\newpage
\microtoc
\newpage

\section{Basics of Group Theory}

\begin{definition}
  Group: A non-empty set $G$ with a binary operation (denoted by $\cdot$) is a group if the following three properties are satisfied:

\begin{itemize}
  \item [] \textbf{(Associativity)}: For all $a,b,c \in G$,  \[(a \cdot b) \cdot c = a \cdot (b \cdot c) \]

  \item [] \textbf{(Identity)}: There exists $e \in G$ s.t. \[a \cdot e = e \cdot a = a \]

  \item [] \textbf{(Inverse)}: For any $a \in G$, there exists $b \in G$ s.t.
    \[a \cdot b = b \cdot a = e \] and $b$ is denoted by $a^{-1}$.

\end{itemize}

\end{definition}


\begin{definition}
  \textbf{Abelian Group}: A commutative group under an operation with the property that for any elements $a,b \in G$, $a \cdot b = b \cdot a$. Examples are $\mathbb{R}^n$ with $+$ and $\mathbb{Z}^n$ with $+$.
\end{definition}


\begin{definition}
  Let $G$ be a group. A subset $H$ in $G$ is a \textbf{subgroup} if $H$ is also a group under the binary operation of $G$, denoted by $H < G$.
\end{definition}

\begin{definition}
  The \textbf{direct product} $G_1 \times G_2$ of groups $G_1$ and $G_2$ is $\{(g_1, g_2) \ | \ g_1 \in G_1, g_2 \in G_2 \}$ with operation 
  \[ (g_1, g_2) \cdot (h_1, h_2) = (g_1 \cdot h_1, g_2 \cdot h_2) \]
\end{definition}


\section{Machine Learning}



% main content







% \lipsum[1-4]

\end{document}
